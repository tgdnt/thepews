\chapter{The zine is here!}

\lettrine[lines=2]{T}{he} Pews is a new vehicle for people
around Trinity to share things with each other in print.

A zine is described as a ``small-circulation self-published
work of original or appropriated texts and images, usually
reproduced via a copy machine.'' Zines have a long history as
a means of communication and self-expression for
underrepresented subcultures. All told, zines have been an
enduring form of grassroots communication. Here, you may share
with your fellow Trinitarians your thoughts that you otherwise
would not have the chance to share.

In this issue, you will find a reflection on joy, written by
Trinity's own Howard Kloepper; my essay about free culture;
and Rev. Loren Drummond's take on the last book of the New
Testament. In the Poems section, poetry by Trinity's own Donna
Ziegenhorn and Rudy Breland (this one selected from his own
previously published zine, and written in 1977) is paired
together with poems by Langston Hughes and Dorothy Parker,
first published in 1926 and recently welcomed into the Public
Domain. Finally, the Stories section features a 1920 science
fiction short story by W. E. B. Du Bois, selected by
Dr. Antonio Byrd in his class on Antiracist Counterstories,
taught at Trinity in the summer of 2021. \emph{The Comet}
truly makes you feel, as if in a dream, what we all have lost.

It has been close to one year since the idea of this zine first
appeared. In the form you hold in your hands, it can be repeated
for as long as there are words to print. I hope you will enjoy
this issue and become inspired to contribute to future issues.

\hspace{20em}---Tiago
