\chapter{The Comet}
\chapterauthor{W. E. B. Du Bois}

\lettrine[lines=2]{H}{e} stood a moment on the steps of the bank, watching the human
river that swirled down Broadway. Few noticed him. Few ever
noticed him save in a way that stung. He was outside the
world---``nothing!'' as he said bitterly. Bits of the words of the
walkers came to him.

``The comet?''

``The comet--''

Everybody was talking of it. Even the president, as he entered,
smiled patronizingly at him, and asked:

``Well, Jim, are you scared?''

``No,'' said the messenger shortly.

``I thought we'd journeyed through the comet's tail once,'' broke
in the junior clerk affably.

``Oh, that was Halley's,'' said the president; ``this is a new
comet, quite a stranger, they say---wonderful, wonderful! I saw it
last night. Oh, by the way, Jim,'' turning again to the
messenger, ``I want you to go down into the lower vaults today.''

The messenger followed the president silently. Of course, they
wanted him to go down to the lower vaults. It was too dangerous
for more valuable men. He smiled grimly and listened.

``Everything of value has been moved out since the water began to
seep in,'' said the president; ``but we miss two volumes of old
records. Suppose you nose around down there,---it isn't very
pleasant, I suppose.''

``Not very,'' said the messenger, as he walked out.

``Well, Jim, the tail of the new comet hits us at noon this
time,'' said the vault clerk, as he passed over the keys; but the
messenger passed silently down the stairs. Down he went beneath
Broadway, where the dim light filtered through the feet of
hurrying men; down to the dark basement beneath; down into the
blackness and silence beneath that lowest cavern. Here with his
dark lantern he groped in the bowels of the earth, under the
world.

He drew a long breath as he threw back the last great iron door
and stepped into the fetid slime within. Here at last was peace,
and he groped moodily forward. A great rat leaped past him and
cobwebs crept across his face. He felt carefully around the
room, shelf by shelf, on the muddied floor, and in crevice and
corner. Nothing. Then he went back to the far end, where somehow
the wall felt different. He sounded and pushed and pried.
Nothing. He started away. Then something brought him back. He
was sounding and working again when suddenly the whole black
wall swung as on mighty hinges, and blackness yawned beyond. He
peered in; it was evidently a secret vault---some hiding place of
the old bank unknown in newer times. He entered hesitatingly. It
was a long, narrow room with shelves, and at the far end, an old
iron chest. On a high shelf lay the two missing volumes of
records, and others. He put them carefully aside and stepped to
the chest. It was old, strong, and rusty. He looked at the vast
and old-fashioned lock and flashed his light on the hinges. They
were deeply incrusted with rust. Looking about, he found a bit
of iron and began to pry. The rust had eaten a hundred years,
and it had gone deep. Slowly, wearily, the old lid lifted, and
with a last, low groan lay bare its treasure---and he saw the dull
sheen of gold!

``Boom!''

A low, grinding, reverberating crash struck upon his ear. He
started up and looked about. All was black and still. He groped
for his light and swung it about him. Then he knew! The great
stone door had swung to. He forgot the gold and looked death
squarely in the face. Then with a sigh he went methodically to
work. The cold sweat stood on his forehead; but he searched,
pounded, pushed, and worked until after what seemed endless
hours his hand struck a cold bit of metal and the great door
swung again harshly on its hinges, and then, striking against
something soft and heavy, stopped. He had just room to squeeze
through. There lay the body of the vault clerk, cold and stiff.
He stared at it, and then felt sick and nauseated. The air
seemed unaccountably foul, with a strong, peculiar odor. He
stepped forward, clutched at the air, and fell fainting across
the corpse.

He awoke with a sense of horror, leaped from the body, and
groped up the stairs, calling to the guard. The watchman sat as
if asleep, with the gate swinging free. With one glance at him
the messenger hurried up to the sub-vault. In vain he called to
the guards. His voice echoed and re-echoed weirdly. Up into the
great basement he rushed. Here another guard lay prostrate on
his face, cold and still. A fear arose in the messenger's heart.
He dashed up to the cellar floor, up into the bank. The
stillness of death lay everywhere and everywhere bowed, bent,
and stretched the silent forms of men. The messenger paused and
glanced about. He was not a man easily moved; but the sight was
appalling! ``Robbery and murder,'' he whispered slowly to himself
as he saw the twisted, oozing mouth of the president where he
lay half-buried on his desk. Then a new thought seized him: If
they found him here alone---with all this money and all these dead
men---what would his life be worth? He glanced about, tiptoed
cautiously to a side door, and again looked behind. Quietly he
turned the latch and stepped out into Wall Street.

How silent the street was! Not a soul was stirring, and yet it
was high-noon---Wall Street? Broadway? He glanced almost wildly up
and down, then across the street, and as he looked, a sickening
horror froze in his limbs. With a choking cry of utter fright he
lunged, leaned giddily against the cold building, and stared
helplessly at the sight.

In the great stone doorway a hundred men and women and children
lay crushed and twisted and jammed, forced into that great,
gaping doorway like refuse in a can---as if in one wild, frantic
rush to safety, they had rushed and ground themselves to death.
Slowly the messenger crept along the walls, wetting his parched
mouth and trying to comprehend, stilling the tremor in his limbs
and the rising terror in his heart. He met a business man,
silk-hatted and frock-coated, who had crept, too, along that
smooth wall and stood now stone dead with wonder written on his
lips. The messenger turned his eyes hastily away and sought the
curb. A woman leaned wearily against the signpost, her head
bowed motionless on her lace and silken bosom. Before her stood
a street car, silent, and within---but the messenger but glanced
and hurried on. A grimy newsboy sat in the gutter with the ``last
edition'' in his uplifted hand: ``Danger!'' screamed its black
headlines. ``Warnings wired around the world. The Comet's tail
sweeps past us at noon. Deadly gases expected. Close doors and
windows. Seek the cellar.'' The messenger read and staggered on.
Far out from a window above, a girl lay with gasping face and
sleevelets on her arms. On a store step sat a little,
sweet-faced girl looking upward toward the skies, and in the
carriage by her lay---but the messenger looked no longer. The
cords gave way---the terror burst in his veins, and with one
great, gasping cry he sprang desperately forward and ran,---ran as
only the frightened run, shrieking and fighting the air until
with one last wail of pain he sank on the grass of Madison
Square and lay prone and still.

When he rose, he gave no glance at the still and silent forms on
the benches, but, going to a fountain, bathed his face; then
hiding himself in a corner away from the drama of death, he
quietly gripped himself and thought the thing through: The comet
had swept the earth and this was the end. Was everybody dead? He
must search and see.

He knew that he must steady himself and keep calm, or he would
go insane. First he must go to a restaurant. He walked up Fifth
Avenue to a famous hostelry and entered its gorgeous,
ghost-haunted halls. He beat back the nausea, and, seizing a
tray from dead hands, hurried into the street and ate
ravenously, hiding to keep out the sights.

``Yesterday, they would not have served me,'' he whispered, as he
forced the food down.

Then he started up the street,---looking, peering, telephoning,
ringing alarms; silent, silent all. Was nobody---nobody---he dared
not think the thought and hurried on.

Suddenly he stopped still. He had forgotten. My God! How could
he have forgotten? He must rush to the subway---then he almost
laughed. No---a car; if he could find a Ford. He saw
one. Gently he lifted off its burden, and took his place on
the seat. He tested the throttle. There was gas. He glided
off, shivering, and drove up the street. Everywhere stood,
leaned, lounged, and lay the dead, in grim and awful
silence. On he ran past an automobile, wrecked and overturned;
past another, filled with a gay party whose smiles yet
lingered on their death-struck lips; on past crowds and groups
of cars, pausing by dead policemen; at 42nd Street he had to
detour to Park Avenue to avoid the dead congestion. He came
back on Fifth Avenue at 57th and flew past the Plaza and by
the park with its hushed babies and silent throng, until as he
was rushing past 72nd Street he heard a sharp cry and saw a
living form leaning wildly out an upper window. He gasped. The
human voice sounded in his ears like the voice of God.

``Hello---hello---help, in God's name!'' wailed the woman. ``There's a
dead girl in here and a man and---and see yonder dead men lying in
the street and dead horses---for the love of God go and bring the
officers---'' And the words trailed off into hysterical tears.

He wheeled the car in a sudden circle, running over the still
body of a child and leaping on the curb. Then he rushed up the
steps and tried the door and rang violently. There was a long
pause, but at last the heavy door swung back. They stared a
moment in silence. She had not noticed before that he was a
Negro. He had not thought of her as white. She was a woman of
perhaps twenty-five---rarely beautiful and richly gowned, with
darkly-golden hair, and jewels. Yesterday, he thought with
bitterness, she would scarcely have looked at him twice. He
would have been dirt beneath her silken feet. She stared at
him.  Of all the sorts of men she had pictured as coming to
her rescue, she had not dreamed of one like him. Not that he
was not human, but he dwelt in a world so far from hers, so
infinitely far, that he seldom even entered her thought. Yet
as she looked at him curiously he seemed quite commonplace and
usual. He was a tall, dark workingman of the better class,
with a sensitive face trained to stolidity and a poor man's
clothes and hands. His face was soft and slow and his manner
at once cold and nervous, like fires long banked, but not out.

So a moment each paused and gauged the other; then the thought
of the dead world without rushed in and they started toward each
other.

``What has happened?'' she cried. ``Tell me! Nothing stirs. All is
silence! I see the dead strewn before my window as winnowed by
the breath of God,---and see---'' She dragged him through great,
silken hangings to where, beneath the sheen of mahogany and
silver, a little French maid lay stretched in quiet, everlasting
sleep, and near her a butler lay prone in his livery.

The tears streamed down the woman's cheeks and she clung to his
arm until the perfume of her breath swept his face and he felt
the tremors racing through her body.

``I had been shut up in my dark room developing pictures of the
comet which I took last night; when I came out---I saw the dead!

``What has happened?'' she cried again.

He answered slowly:

``Something---comet or devil---swept across the earth this morning
and---many are dead!''

``Many? Very many?''

``I have searched and I have seen no other living soul but you.''

She gasped and they stared at each other.

``My---father!'' she whispered.

``Where is he?''

``He started for the office.''

``Where is it?''

``In the Metropolitan Tower.''

``Leave a note for him here and come.''

Then he stopped.

``No,'' he said firmly---``first, we must go---to Harlem.''

``Harlem!'' she cried. Then she understood. She tapped her foot at
first impatiently. She looked back and shuddered. Then she came
resolutely down the steps.

``There's a swifter car in the garage in the court,'' she said.

``I don't know how to drive it,'' he said.

``I do,'' she answered.

In ten minutes they were flying to Harlem on the wind. The Stutz
rose and raced like an airplane. They took the turn at 110th
Street on two wheels and slipped with a shriek into 135th.

He was gone but a moment. Then he returned, and his face was
gray. She did not look, but said:

``You have lost---somebody?''

``I have lost---everybody,'' he said, simply---``unless---''

He ran back and was gone several minutes---hours they seemed to
her.

``Everybody,'' he said, and he walked slowly back with something
film-like in his hand which he stuffed into his pocket.

``I'm afraid I was selfish,'' he said. But already the car was
moving toward the park among the dark and lined dead of
Harlem---the brown, still faces, the knotted hands, the homely
garments, and the silence---the wild and haunting silence. Out of
the park, and down Fifth Avenue they whirled. In and out among
the dead they slipped and quivered, needing no sound of bell or
horn, until the great, square Metropolitan Tower hove in sight.
Gently he laid the dead elevator boy aside; the car shot upward.
The door of the office stood open. On the threshold lay the
stenographer, and, staring at her, sat the dead clerk. The inner
office was empty, but a note lay on the desk, folded and
addressed but unsent:

\hspace{2em}Dear Daughter:

\hspace{2em}I've gone for a hundred mile spin in Fred's new Mercedes. Shall
not be back before dinner. I'll bring Fred with me.

\hspace{2em}J.B.H.

``Come,'' she cried nervously. ``We must search the city.''

Up and down, over and across, back again---on went that ghostly
search. Everywhere was silence and death---death and silence! They
hunted from Madison Square to Spuyten Duyvel; they rushed across
the Williamsburg Bridge; they swept over Brooklyn; from the
Battery and Morningside Heights, they scanned the river. Silence,
silence everywhere, and no human sign. Haggard and bedraggled
they puffed a third time slowly down Broadway, under the
broiling sun, and at last stopped. He sniffed the air. An odor---a
smell---and with the shifting breeze a sickening stench filled
their nostrils and brought its awful warning. The girl settled
back helplessly in her seat.

``What can we do?'' she cried.

It was his turn now to take the lead, and he did it quickly.

``The long distance telephone---the telegraph and the cable---night
rockets and then---flight!''

She looked at him now with strength and confidence. He did not
look like men, as she had always pictured men; but he acted like
one and she was content. In fifteen minutes they were at the
central telephone exchange. As they came to the door, he stepped
quickly before her and pressed her gently back as he closed it.
She heard him moving to and fro and knew his burdens---the poor,
little burdens he bore. When she entered, he was alone in the
room. The grim switchboard flashed its metallic face in cryptic,
sphinx-like immobility. She seated herself on a stool and donned
the bright earpiece. She looked at the mouthpiece. She had never
looked at one so closely before. It was wide and black, pimpled
with usage; inert; dead; almost sarcastic in its unfeeling
curves. It looked---she beat back the thought---but it looked,---it
persisted in looking like---she turned her head and found herself
alone. One moment she was terrified; then she thanked him
silently for his delicacy and turned resolutely, with a quick
intaking of breath.

``Hello!'' she called in low tones. She was calling to the world.
The world must answer. Would the world answer? Was the world---

Silence!

She had spoken too low.

``Hello!'' she cried, full-voiced.

She listened. Silence! Her heart beat quickly. She cried in
clear, distinct, loud tones: ``Hello---hello---hello!''

What was that whirring? Surely---no---was it the click of a
receiver?

She bent close, she moved the pegs in the holes, and called and
called, until her voice rose almost to a shriek, and her heart
hammered. It was as if she had heard the last flicker of
creation, and the evil was silence. Her voice dropped to a sob.
She sat stupidly staring into the black and sarcastic
mouthpiece, and the thought came again. Hope lay dead within
her. Yes, the cable and the rockets remained; but the world---she
could not frame the thought or say the word. It was too
mighty---too terrible! She turned toward the door with a new fear
in her heart. For the first time she seemed to realize that she
was alone in the world with a stranger, with something more than
a stranger,---with a man alien in blood and culture---unknown,
perhaps unknowable. It was awful! She must escape---she must fly;
he must not see her again. Who knew what awful thoughts---

She gathered her silken skirts deftly about her young, smooth
limbs---listened, and glided into a sidehall. A moment she shrank
back: the hall lay filled with dead women; then she leaped to
the door and tore at it, with bleeding fingers, until it swung
wide. She looked out. He was standing at the top of the
alley,---silhouetted, tall and black, motionless. Was he looking
at her or away? She did not know---she did not care. She simply
leaped and ran---ran until she found herself alone amid the dead
and the tall ramparts of towering buildings.

She stopped. She was alone. Alone! Alone on the streets---alone in
the city---perhaps alone in the world! There crept in upon her the
sense of deception---of creeping hands behind her back---of silent,
moving things she could not see,---of voices hushed in fearsome
conspiracy. She looked behind and sideways, started at strange
sounds and heard still stranger, until every nerve within her
stood sharp and quivering, stretched to scream at the barest
touch. She whirled and flew back, whimpering like a child, until
she found that narrow alley again and the dark, silent figure
silhouetted at the top. She stopped and rested; then she walked
silently toward him, looked at him timidly; but he said nothing
as he handed her into the car. Her voice caught as she
whispered:

``Not---that.''

And he answered slowly: ``No---not that!''

They climbed into the car. She bent forward on the wheel and
sobbed, with great, dry, quivering sobs, as they flew toward the
cable office on the east side, leaving the world of wealth and
prosperity for the world of poverty and work. In the world
behind them were death and silence, grave and grim, almost
cynical, but always decent; here it was hideous. It clothed
itself in every ghastly form of terror, struggle, hate, and
suffering. It lay wreathed in crime and squalor, greed and lust.
Only in its dread and awful silence was it like to death
everywhere.

Yet as the two, flying and alone, looked upon the horror of the
world, slowly, gradually, the sense of all-enveloping death
deserted them. They seemed to move in a world silent and
asleep,---not dead. They moved in quiet reverence, lest somehow
they wake these sleeping forms who had, at last, found peace.
They moved in some solemn, world-wide Friedhof, above which some
mighty arm had waved its magic wand. All nature slept
until---until, and quick with the same startling thought, they
looked into each other's eyes---he, ashen, and she, crimson, with
unspoken thought. To both, the vision of a mighty beauty---of
vast, unspoken things, swelled in their souls; but they put it
away.

Great, dark coils of wire came up from the earth and down from
the sun and entered this low lair of witchery. The gathered
lightnings of the world centered here, binding with beams of
light the ends of the earth. The doors gaped on the gloom
within. He paused on the threshold.

``Do you know the code?'' she asked.

``I know the call for help---we used it formerly at the bank.''

She hardly heard. She heard the lapping of the waters far
below,---the dark and restless waters---the cold and luring waters,
as they called. He stepped within. Slowly she walked to the
wall, where the water called below, and stood and waited. Long
she waited, and he did not come. Then with a start she saw him,
too, standing beside the black waters. Slowly he removed his
coat and stood there silently. She walked quickly to him and
laid her hand on his arm. He did not start or look. The waters
lapped on in luring, deadly rhythm. He pointed down to the
waters, and said quietly:

``The world lies beneath the waters now---may I go?''

She looked into his stricken, tired face, and a great pity
surged within her heart. She answered in a voice clear and calm,
``No.''

Upward they turned toward life again, and he seized the wheel.
The world was darkening to twilight, and a great, gray pall was
falling mercifully and gently on the sleeping dead. The ghastly
glare of reality seemed replaced with the dream of some vast
romance. The girl lay silently back, as the motor whizzed along,
and looked half-consciously for the elf-queen to wave life into
this dead world again. She forgot to wonder at the quickness
with which he had learned to drive her car. It seemed natural.
And then as they whirled and swung into Madison Square and at
the door of the Metropolitan Tower she gave a low cry, and her
eyes were great! Perhaps she had seen the elf-queen?

The man led her to the elevator of the tower and deftly they
ascended. In her father's office they gathered rugs and chairs,
and he wrote a note and laid it on the desk; then they ascended
to the roof and he made her comfortable. For a while she rested
and sank to dreamy somnolence, watching the worlds above and
wondering. Below lay the dark shadows of the city and afar was
the shining of the sea. She glanced at him timidly as he set
food before her and took a shawl and wound her in it, touching
her reverently, yet tenderly. She looked up at him with
thankfulness in her eyes, eating what he served. He watched the
city. She watched him. He seemed very human,---very near now.

``Have you had to work hard?'' she asked softly.

``Always,'' he said.

``I have always been idle,'' she said. ``I was rich.''

``I was poor,'' he almost echoed.

``The rich and the poor are met together,'' she began, and he
finished:

``The Lord is the Maker of them all.''

``Yes,'' she said slowly; ``and how foolish our human distinctions
seem---now,'' looking down to the great dead city stretched below,
swimming in unlightened shadows.

``Yes---I was not---human, yesterday,'' he said.

She looked at him. ``And your people were not my people,'' she
said; ``but today---'' She paused. He was a man,---no more; but he
was in some larger sense a gentleman,---sensitive, kindly,
chivalrous, everything save his hands and---his face. Yet
yesterday---

``Death, the leveler!'' he muttered.

``And the revealer,'' she whispered gently, rising to her feet
with great eyes. He turned away, and after fumbling a moment
sent a rocket into the darkening air. It arose, shrieked, and
flew up, a slim path of light, and scattering its stars abroad,
dropped on the city below. She scarcely noticed it. A vision of
the world had risen before her. Slowly the mighty prophecy of
her destiny overwhelmed her. Above the dead past hovered the
Angel of Annunciation. She was no mere woman. She was neither
high nor low, white nor black, rich nor poor. She was primal
woman; mighty mother of all men to come and Bride of Life. She
looked upon the man beside her and forgot all else but his
manhood, his strong, vigorous manhood---his sorrow and sacrifice.
She saw him glorified. He was no longer a thing apart, a
creature below, a strange outcast of another clime and blood,
but her Brother Humanity incarnate, Son of God and great
All-Father of the race to be.

He did not glimpse the glory in her eyes, but stood looking
outward toward the sea and sending rocket after rocket into the
unanswering darkness. Dark-purple clouds lay banked and billowed
in the west. Behind them and all around, the heavens glowed in
dim, weird radiance that suffused the darkening world and made
almost a minor music. Suddenly, as though gathered back in some
vast hand, the great cloud-curtain fell away. Low on the horizon
lay a long, white star---mystic, wonderful! And from it fled
upward to the pole, like some wan bridal veil, a pale, wide
sheet of flame that lighted all the world and dimmed the stars.

In fascinated silence the man gazed at the heavens and dropped
his rockets to the floor. Memories of memories stirred to life
in the dead recesses of his mind. The shackles seemed to
rattle and fall from his soul. Up from the crass and crushing
and cringing of his caste leaped the lone majesty of kings
long dead. He arose within the shadows, tall, straight, and
stern, with power in his eyes and ghostly scepters hovering to
his grasp. It was as though some mighty Pharaoh lived again,
or curled Assyrian lord. He turned and looked upon the lady
and found her gazing straight at him.

Silently, immovably, they saw each other face to face---eye to
eye. Their souls lay naked to the night. It was not lust; it was
not love---it was some vaster, mightier thing that needed neither
touch of body nor thrill of soul. It was a thought divine,
splendid.

Slowly, noiselessly, they moved toward each other---the heavens
above, the seas around, the city grim and dead below. He loomed
from out the velvet shadows vast and dark. Pearl-white and
slender, she shone beneath the stars. She stretched her jeweled
hands abroad. He lifted up his mighty arms, and they cried each
to the other, almost with one voice, ``The world is dead.''

``Long live the---''

``Honk! Honk!'' Hoarse and sharp the cry of a motor drifted
clearly up from the silence below. They started backward with a
cry and gazed upon each other with eyes that faltered and fell,
with blood that boiled.

``Honk! Honk! Honk! Honk!'' came the mad cry again, and almost
from their feet a rocket blazed into the air and scattered its
stars upon them. She covered her eyes with her hands, and her
shoulders heaved. He dropped and bowed, groped blindly on his
knees about the floor. A blue flame spluttered lazily after an
age, and she heard the scream of an answering rocket as it flew.

Then they stood still as death, looking to opposite ends of the
earth.

``Clang---crash---clang!''

The roar and ring of swift elevators shooting upward from below
made the great tower tremble. A murmur and babel of voices swept
in upon the night. All over the once dead city the lights
blinked, flickered, and flamed; and then with a sudden clanging
of doors the entrance to the platform was filled with men, and
one with white and flying hair rushed to the girl and lifted her
to his breast. ``My daughter!'' he sobbed.

Behind him hurried a younger, comelier man, carefully clad in
motor costume, who bent above the girl with passionate
solicitude and gazed into her staring eyes until they narrowed
and dropped and her face flushed deeper and deeper crimson.

``Julia,'' he whispered; ``my darling, I thought you were gone
forever.''

She looked up at him with strange, searching eyes.

``Fred,'' she murmured, almost vaguely, ``is the world---gone?''

``Only New York,'' he answered; ``it is terrible---awful! You
know,---but you, how did you escape---how have you endured this
horror? Are you well? Unharmed?''

``Unharmed!'' she said.

``And this man here?'' he asked, encircling her drooping form with
one arm and turning toward the Negro. Suddenly he stiffened and
his hand flew to his hip. ``Why!'' he snarled.
``It's a---nigger---Julia! Has he---has he dared''

She lifted her head and looked at her late companion curiously
and then dropped her eyes with a sigh.

``He has dared---all, to rescue me,'' she said quietly, ``and I---thank
him---much.'' But she did not look at him again. As the couple
turned away, the father drew a roll of bills from his pockets.

``Here, my good fellow,'' he said, thrusting the money into the
man's hands, ``take that,---what's your name?''

``Jim Davis,'' came the answer, hollow-voiced.

``Well, Jim, I thank you. I've always liked your people. If you
ever want a job, call on me.'' And they were gone.

The crowd poured up and out of the elevators, talking and
whispering.

``Who was it?''

``Are they alive?''

``How many?''

``Two!''

``Who was saved?''

``A white girl and a nigger---there she goes.''

``A nigger? Where is he? Let's lynch the damned---''

``Shut up---he's all right-he saved her.''

``Saved hell! He had no business---''

``Here he comes.''

Into the glare of the electric lights the colored man moved
slowly, with the eyes of those that walk and sleep.

``Well, what do you think of that?'' cried a bystander; ``of all
New York, just a white girl and a nigger!''

The colored man heard nothing. He stood silently beneath the
glare of the light, gazing at the money in his hand and
shrinking as he gazed; slowly he put his other hand into his
pocket and brought out a baby's filmy cap, and gazed again. A
woman mounted to the platform and looked about, shading her
eyes. She was brown, small, and toil-worn, and in one arm lay
the corpse of a dark baby. The crowd parted and her eyes fell on
the colored man; with a cry she tottered toward him.

``Jim!''

He whirled and, with a sob of joy, caught her in his arms.